%% 
%% Copyright 2007-2020 Elsevier Ltd
%% 
%% This file is part of the 'Elsarticle Bundle'.
%% ---------------------------------------------
%% 
%% It may be distributed under the conditions of the LaTeX Project Public
%% License, either version 1.2 of this license or (at your option) any
%% later version.  The latest version of this license is in
%%    http://www.latex-project.org/lppl.txt
%% and version 1.2 or later is part of all distributions of LaTeX
%% version 1999/12/01 or later.
%% 
%% The list of all files belonging to the 'Elsarticle Bundle' is
%% given in the file `manifest.txt'.
%% 
%% Template article for Elsevier's document class `elsarticle'
%% with harvard style bibliographic references

\documentclass[preprint,12pt]{elsarticle}

%% Use the option review to obtain double line spacing
%% \documentclass[preprint,review,12pt]{elsarticle}

%% Use the options 1p,twocolumn; 3p; 3p,twocolumn; 5p; or 5p,twocolumn
%% for a journal layout:
%% \documentclass[final,1p,times]{elsarticle}
%% \documentclass[final,1p,times,twocolumn]{elsarticle}
%% \documentclass[final,3p,times]{elsarticle}
%% \documentclass[final,3p,times,twocolumn]{elsarticle}
%% \documentclass[final,5p,times]{elsarticle}
%% \documentclass[final,5p,times,twocolumn]{elsarticle}

%% For including figures, graphicx.sty has been loaded in
%% elsarticle.cls. If you prefer to use the old commands
%% please give \usepackage{epsfig}

%% The amssymb package provides various useful mathematical symbols
\usepackage{amssymb}

\usepackage{amsmath}

\usepackage{amsfonts}
\usepackage{graphicx}
\newcommand{\Rho}{\mathrm{\textit{P}}}
\usepackage{subcaption}
\usepackage{csquotes}
\usepackage{algorithm} 
\usepackage{algpseudocode} 
\usepackage{longtable}
%\usepackage[top=1.0in, bottom=1.0in, left=1.0in, right=1.0in]{geometry}
\usepackage{url}
\def\UrlBreaks{\do\/\do-}
\usepackage{breakurl}
\usepackage[breaklinks]{hyperref}
%% The amsthm package provides extended theorem environments
%% \usepackage{amsthm}

%% The lineno packages adds line numbers. Start line numbering with
%% \begin{linenumbers}, end it with \end{linenumbers}. Or switch it on
%% for the whole article with \linenumbers.
%% \usepackage{lineno}

\journal{Information Sciences}

\begin{document}

\begin{frontmatter}

%% Title, authors and addresses

%% use the tnoteref command within \title for footnotes;
%% use the tnotetext command for theassociated footnote;
%% use the fnref command within \author or \address for footnotes;
%% use the fntext command for theassociated footnote;
%% use the corref command within \author for corresponding author footnotes;
%% use the cortext command for theassociated footnote;
%% use the ead command for the email address,
%% and the form \ead[url] for the home page:
%% \title{Title\tnoteref{label1}}
%% \tnotetext[label1]{}
%% \author{Name\corref{cor1}\fnref{label2}}
%% \ead{email address}
%% \ead[url]{home page}
%% \fntext[label2]{}
%% \cortext[cor1]{}
%% \affiliation{organization={},
%%             addressline={},
%%             city={},
%%             postcode={},
%%             state={},
%%             country={}}
%% \fntext[label3]{}

\title{A Ranked Solution for Social Media Fact Checking Using Epidemic Spread}

%% use optional labels to link authors explicitly to addresses:
%% \author[label1,label2]{}
%% \affiliation[label1]{organization={},
%%             addressline={},
%%             city={},
%%             postcode={},
%%             state={},
%%             country={}}
%%
%% \affiliation[label2]{organization={},
%%             addressline={},
%%             city={},
%%             postcode={},
%%             state={},
%%             country={}}

\author[inst1]{John Hawthorne Smith}

\affiliation[inst1]
        {organization={Northwestern University},
        addressline={633 Clark St},
        city={Evanston}, postcode={60208},
        state={IL},
        country={United States}}

\author[inst2]{Author Two}

\begin{abstract}
%% Text of abstract
Within the past decade, social media has become a primary platform for consumption of information and current events. Unlike with traditional news sources, however, social media posts do not have to go through a rigorous validation process prior to publication. The 2019 Mueller Report illustrates how malicious actors have taken advantage of these lax requirements to sway public opinion on topics from the \#blacklivesmatter movement to the 2016 U.S. Presidential election. 

Currently, social media companies rely primarily on communal-policing of misinformation: this is unlikely that this will happen with regularity. To counteract this, other literature on the topic is focused on building neural networks that will be able to detect accurate from misleading content; however, the rapidly evolving nature of misinformation means that they will have to be retrained and redeployed on an iterative and time-consuming basis.

This thesis, therefore, proposes a novel approach to the problem: treating misinformation as a virus. This thesis proposes a ranking system that third-party fact checkers can utilize to prioritize posts for checking. This algorithm is then tested against multiple data sets with strong positive results, decreasing viral spread in a matter of minutes.
\end{abstract}


\begin{keyword}
%% keywords here, in the form: keyword \sep keyword
Social media \sep Fake news \sep Misinformation \sep Partisanship \sep Networks
\end{keyword}

\end{frontmatter}

%% \linenumbers

%% main text
\section{Introduction}
\label{Introduction}

%% For citations use: 
%%       \citet{<label>} ==> Jones et al. [21]
%%       \citep{<label>} ==> [21]
%%

In 2018, social media overtook print media as the fourth most popular source of news for Americans. As of 2019, 20\% ``often" got their news from social media \cite{shearer2018social}, while 68\% of all American got their news from social media at least occasionally \cite{matsa2018news}. This is not inherently a bad thing - in countries with authoritarian leaders and state-run media, social media may be the only outlet for opposition spokespeople to share their messages \cite{walker2014breaking}; ordinary citizens can now contribute their stories and experiences without the high financial barrier to entry that traditional journalism requires \cite{qualman2012socialnomics, tapscott2008wikinomics}; in unmanageable situations and crises, such as the 2017 Manchester bombing, social media allows for the instantaneous exchange of information between individuals and emergency management agencies \cite{mirbabaie2020breaking, eriksson2016facebook}; in situations like the COVID-19 pandemic, where information is rapidly evolving, social media allows for immediate knowledge dissemination by dramatically shortening the traditional time from publication to widespread translation to adoption \cite{chan2020social}. 
 
However, not all information shared on social media is true, and the subset of people who primarily get their news from social media tend to be less engaged, less knowledgeable of current events, more likely to hear unproven claims and conspiracy theories than those who get their news from more traditional sources, and are more likely to believe these conspiracies \cite{mitchell2020americans}. In just two examples of this, Edgar Maddison Welch drove from North Carolina to Washington D.C. with an AR-15 to investigate a fake pedophile conspiracy ring known as ``pizzagate" in December 2016, \cite{goldman2016comet}; in 2018, Burmese officials created over 1,000 Facebook posts filled with hate speech and detailing fake crimes committed by the Rohynga Muslim minority to justify one of the largest forced migrations in recent history \cite{subedar2018country}.

Exacerbating this problem, the spread of correct information is much slower than that of misinformation: while true rumors tend to be validated within 2 hours of the first tweet, false rumors take about 14 hours to be debunked \cite{zubiaga2016analysing,shao2016hoaxy}. This is a problem since, in a crisis event, 50\% of all retweets occur within the first hour after a tweet is shared and 75\% are within the first day \cite{kwak2010twitter}. Even if an untrue story is debunked, corrective information does not reach as broad of an audience as misinformation does \cite{maddock2015characterizing, vosoughi2018spread}, and, in some cases, rumors and other fake stories actually see an increase in spread \textbf{after} they have been debunked \cite{starbird2014rumors}. In fact, this fits a pattern where highly controversial and politicized topics spark backfire effects when passionately held misconceptions are challenged \cite{gollust2009polarizing,nyhan2010corrections,nyhan2013hazards,redlawsk2010affective,schaffner2016misinformation,hart2012boomerang}.

Therefore, an alternative solution is necessary that 


\newpage
\bibliographystyle{elsarticle-num-names} 
\bibliography{bibliography}
\end{document}


